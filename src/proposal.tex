\chapter{Proposal: Link Management For Quantum Network}
\label{proposal}

\section{Overview}

This chapter proposes the protocol for the management of the physical bell pairs that are available on each quantum link.

This protocol has two separated phases. The first phase is the negotiation about the set of RuleSets that are going to be established in the next round. The second phase is the negotiation about when to apply the next set of RuleSets.
These negotiations will take place between the two end point of each link.

\section{Assumptions}

This protocol is proposed based on the following assumptions.

\begin{itemize}
  \item Existence of both a classical link and quantum link between two neighboring nodes.
  \item No delay in transmitting messages 
  \item No failures in nodes and links
  \item Transmission of classical messages each of which includes a sequence number (an incremental identifier) from a support node every time a new bell pair is generated.
\end{itemize}

\section{Requirements}

This protocol has several requirements as follows.

\subsubsection{Functional requirements}

\begin{itemize}
  \item The two end nodes of each link must coordinate what set of RuleSets and their order to execute in the next round
  \item The two end nodes of each link must coordinate the timing of when to execute these RuleSets
\end{itemize}

\subsubsection{Non functional requirements}

\begin{itemize}
  \item This protocol must be adaptable to any link architecture assumed a quantum channel is based on.
\end{itemize}

\section{Link Allocation Policy}
\section{The Negotiation About The Upcoming Link Allocation}
\section{The Negotiation About The Timing of Updating Link Allocation}
\section{Resource Allocation }
\section{Link Management Finite State Machines}
\section{Link Management Finite State Events}
\section{Type of Messages}
\section{Relationship With Connection Setup}
\section{Relationship With Connection Teardown}

%%% Local Variables:
%%% mode: japanese-latex
%%% TeX-master: "../bthesis"
%%% End:
