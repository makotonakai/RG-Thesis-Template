\chapter{Proposal: Link Management For Quantum Network}
\label{proposal}

\section{Overview}

This chapter proposes the protocol for the management of the link-level Bell pairs.

This protocol has two separated phases. The first phase is the negotiation about the set of RuleSets that are going to be established in the next round. The second phase is the negotiation about when to apply the next set of RuleSets.
These negotiations will take place between the two end point of each link.

\subsection{Link Allocation Policy}

In order to establish multiple connections over a single link, both end nodes of the link need to make the coordinated decisions about what connections need to be established.
This set of connections, to be more specific, the set of RuleSets, is called \textbf{Link Allocation Policy} in the rest of this thesis.

\subsection{Link Allocation Policy Negotiation Phase}

After the node receives a message that notifies the establishment of a new connection, or the termination of one of the existing connections, both nodes on a single link need to agree on the link allocation policy that is going to be executed in the next round.
Therefore, this protocol involves the transmission of messages that include the information of the next link allocation policy in each node.  It has to be mentioned that the order of arrival of RuleSets in the next policy might be different, so the protocol also requires a mechanism to determine which policy needs to be prioritized.
This can be achieved by inserting a random integer to the message and adopting the order with the larger value.

\subsection{The Timing Negotiation Phase}

The end nodes of a physical link also need to align the timing of updating the link policy in order to assign the same Bell pair to the connection.
Otherwise,they might allocate the physical qubits of a different Bell pair to two different connections, which might end up with the failure of the entire connection.

\subsection{Resource Management}

The actual resource allocation process needs to take place before or during the execution of the RuleSets that were determined in the previous steps.
In contrary, the release of physical resources that were allocated to the terminated RuleSets need to be executed after the notification of connection teardown, which the node receives from the networking layer.

\section{Messages}

This protocol involves the exchange of two kinds of messages, which are \textbf{LinkAllocationUpdateMessage} and \textbf{BarrierMessage}.
This section proposes the required fields and their types in each message.

\subsection{LinkAllocationUpdateMessage}
 
This message contains the fields shown in table \ref{fig:lau-message}.

\begin{table}[ht]
  \begin{center}
    \begin{tabular}{|m{10em}|m{10em}|m{10em}|} \hline
      Field Name & Type & Explanation \\ \hline \cline{1-3}
      srcAddress & integer & The source address \\ \cline{1-3}
      destAddress & integer & The destination address \\ \cline{1-3}
      activeLinkAllocations & unsigned long [] & The array of IDs of the RuleSets in the active link allocation policy \\ \cline{1-3}
      nextLinkAllocations & unsigned long [] & The array of IDs of the RuleSets in the upcoming link allocation policy \\ \cline{1-3}
      randomValue & integer & A random value \\ \cline{1-3}
    \end{tabular}
    \caption{The Message Fields in a LinkAllocationUpdateMessage}
    \label{fig:lau-message}
  \end{center}
\end{table}

\subsection{RejectLinkAllocationUpdateMessage}
 
This message contains the fields shown in table \ref{fig:reject-lau-message}.

\begin{table}[ht]
  \begin{center}
    \begin{tabular}{|m{10em}|m{10em}|m{10em}|} \hline
      Field Name & Type & Explanation \\ \hline \cline{1-3}
      srcAddress & integer & The source address \\ \cline{1-3}
      destAddress & integer & The destination address \\ \cline{1-3}
    \end{tabular}
    \caption{The Message Fields in a RejectLinkAllocationUpdateMessage}
    \label{fig:reject-lau-message}
  \end{center}
\end{table}

\subsection{BarrierMessage}

This message contains the fields shown in table \ref{fig:barrier-message}.

\begin{table}[ht]
  \begin{center}
    \begin{tabular}{|c|c|c|} \hline
      Field Name & Type & Explanation \\ \hline \cline{1-3}
      srcAddress & integer & The source address \\ \cline{1-3}
      destAddress & integer & The destination address \\ \cline{1-3}
      sequenceNumber & integer & A sequence number of the first available physical Bell Pair \\ \cline{1-3}
    \end{tabular}
    \caption{The Message Fields in a BarrierMessage}
    \label{fig:barrier-message}
  \end{center}
\end{table}

\newpage

\section{Finite State Machine For Link Allocation Policy}

A finite state machine (FSM) is commonly provides a simple and clear description about the behavior of the communication protocol \cite{BOCHMANN1978361}.
Each state in the finite state machine represents the condition of a communication node, its events represents the change such as transmission and reception of messages, and the action represents the reaction to the event based on the previous condition.
Each event triggers a state transition.
This section explains the behavior of one of the end nodes of a link during the negotiation phase. 

\subsection{States}

This subsection introduces all the states that a node can be during the negotiations related to the next link allocation policy.
\subsubsection{Init}
This is the initial state that each node starts with.  
In this state, neither the negotiation about the upcoming link allocation policy or the one about when to update the policy are happening.
The FSM transits into either IncomingLAUWait state or LAUNotSent state by sending an LinkAllocationUpdateMessage or receiving the one from its neighboring nodes.

\subsubsection{IncomingLAUWait}
This is the state after a node sends a LinkAllocationUpdateMessage to its neighboring node. 
In this state, a node is waiting for the incoming LinkAllocationUpdateMessage from those nodes in return, 
so that the FSM can move to SyncNextPolicy state by coordinating the new link allocation policy.

\subsubsection{LAUNotSent}
This is the state when a node receives a LinkAllocationUpdateMessage from its neighboring node. 
In this state, a node is about to send LinkAllocationUpdateMessages back to those nodes, so that the FSM can move to SyncNextPolicy state by coordinating the new link allocation policy.

\subsubsection{SyncNextPolicy}
This is the state when both end nodes coordinated the next link allocation policy.
The FSM transits into either BarrierNotSent state or IncomingBarrierWait state if the negotiation goes successfully, otherwise it transits back to Init if they fail.

\subsubsection{IncomingBarrierWait}
This is the state after a node sends a BarrierMessage to its neighboring node. 
In this state, a node is waiting for the incoming BarrierMessage from that node in return, 
so that the FSM can move to BarrierMessage state by coordinating from which Bell Pair the new link allocation policy should be applied.

\subsubsection{BarrierNotSent}
This is the state when a node receives a BarrierMessage from its neighboring node. 
In this state, a node is about to send BarrierMessage back to that node, 
so that the FSM can move to SyncNextSeqNum state by coordinating from which Bell Pair the new link allocation policy should be applied.

\subsubsection{SyncNextSeqNum}
This is the state when both end nodes of a link successfully coordinated from which Bell Pair the new link allocation policy will be updated.
The FSM transits to the Init state until the next negotiation about the link allocation policy becomes triggered from the networking layer.

\subsection{Events}

This subsection introduces all the events that occur during the negotiation of the upcoming link allocation policy.

\subsubsection{TxLAU}
This event indicates the transmission of a LinkAllocationUpdateMessage.

\subsubsection{RxLAU}
This event indicates the reception of a LinkAllocationUpdateMessage.

\subsubsection{TxBr}
This event indicates the transmission of a BarrierMessage.

\subsubsection{RxBr}
This event indicates the reception of a BarrierMessage.

\subsubsection{LAUSuccess}
This event indicates the success in the coordination of the next link allocation policy.

\subsubsection{LAUFail}
This event indicates the failure in the coordination of the next link allocation policy.


\subsubsection{BarrierSuccess}
This event indicates the success in the coordination of the first sequence number that the new link allocation policy is applied.

\subsubsection{BarrierFail}
This event indicates the failure in the coordination of the first sequence number that the new link allocation policy is applied.

\newpage

\subsection{Description of Finite State Machine}

Fig \ref{fig:state-machine-lau} shows the finite state machine that describes the change of states during the negotiations related to the next link allocation policy.

\begin{figure}[ht] % ’ht’ tells LaTeX to place the figure ’here’ or at the top of the page
  \centering % centers the figure
  \begin{tikzpicture}[shorten >=1pt,node distance=6cm,on grid,auto]
    \node[state, initial] (Init) {Init};
    \node[state, below left of=Init] (IncomingLAUWait) {IncomingLAUWait};
    \node[state, right of=IncomingLAUWait, xshift=1cm] (LAUNotSent) {LAUNotSent};
    \node[state, below right of=IncomingLAUWait] (SyncNextPolicy) {SyncNextPolicy};
    \node[state, below left of=SyncNextPolicy] (IncomingBarrierWait) {IncomingBarrierWait};
    \node[state, right of=IncomingBarrierWait, xshift=1cm] (BarrierNotSent) {BarrierNotSent};
    \node[state, accepting, below right of=IncomingBarrierWait] (SyncNextSeqNum) {SyncNextSeqNum};

   \draw[->] (Init) edge [bend right] node {\tt TxLAU} (IncomingLAUWait);
   \draw[->] (Init) edge [bend left] node {\tt RxLAU} (LAUNotSent);
   \draw[->] (IncomingLAUWait) edge [bend right] node {\tt RxLAU} (SyncNextPolicy);
   \draw[->] (LAUNotSent) edge [bend left] node {\tt TxLAU} (SyncNextPolicy);
   \draw[->] (SyncNextPolicy) edge [bend right] node {\tt TxBr} (IncomingBarrierWait);
   \draw[->] (SyncNextPolicy) edge [bend left] node {\tt RxBr} (BarrierNotSent);
   \draw[->] (SyncNextPolicy) edge [loop left] node {\tt LAUSuccess} ();
   \draw[->] (SyncNextPolicy) edge [bend right] node {\tt LAUFail} (Init);
   \draw[->] (IncomingBarrierWait) edge [bend right] node {\tt RxBr} (SyncNextSeqNum);
   \draw[->] (BarrierNotSent) edge [bend left] node {\tt TxBr} (SyncNextSeqNum);
   \draw[->] (SyncNextSeqNum) edge [loop left] node {\tt BarrierSuccess} ();
   \draw[->] (SyncNextSeqNum) edge [bend right] node {\tt BarrierFail} (Init);

  \end{tikzpicture}
  \caption{The FSM for the negotiation phase}
  \label{fig:state-machine-lau}
\end{figure}

\section{Finite State Machine For Link Management}

This section provides the different finite state machine for the link management phase. It focuses on the behavior of a link-level Bell pair for simplicity.

\subsection{States}

This subsection introduces all the states that a link-level Bell pair can be during the resource management.

\subsubsection{Up}
This is the state when a link-level Bell pair between its two end nodes. In this state, it is not allocated to any specific RuleSet.
The FSM transits to Allocated if the link becomes allocated to one of the RuleSets in the active link allocation policy.

\subsubsection{Down}
This is the state when a physical Bell pair on a quantum link is not established between its two end nodes. In this state, it can be no Bell pair between two existing quantum memories in the case of a quantum repeaters with those memories, or the situation when incoming photons have not arrived to memoryless quantum repeaters.
The FSM transits to Up if a Bell pair is established.

\subsubsection{Allocated}
This is the state when a physical Bell pair is allocated to one of the RuleSets in the active link allocation policy.
The FSM transits to Up if the link becomes released after the connection that this link used to be allocated is terminated.
It can also transits to Down if the physical qubits on the link are measured during execution of the RuleSet that this link is allocated to.

\subsection{Events}

This subsection introduces all the events that cause transition of states.

\subsubsection{BellPairGen}
This event indicates the generation of a Bell pair

\subsubsection{Allocate}
This event indicates the allocation of a given Bell pair to RuleSet.

\subsubsection{Free}
This event indicates the release of an available Bell pair from that RuleSet that it is used to be allocated to.

\subsubsection{Measure}
This event indicates the measurement of two physical qubits of the given Bell pair while the RuleSet is being executed.

\subsection{Description of Finite State Machine}

Fig \ref{fig:link-state-machine} is the state machine that describes the transition for the state of a single link-level Bell pair.

\begin{figure}[ht] % ’ht’ tells LaTeX to place the figure ’here’ or at the top of the page
  \centering % centers the figure
  \begin{tikzpicture}[shorten >=1pt,node distance=3cm,on grid,auto]
    \node[state] (Up) {Up};
    \node[state, initial, accepting, below = 2cm of Up] (Down) {Down};
    \node[state, right = 3cm of Up] (Allocated) {Allocated};

    \draw[->] (Down) edge [bend left] node {\tt BellPairGen} (Up);
    \draw[->] (Up) edge [bend left] node {\tt Allocate} (Allocated);
    \draw[->] (Allocated) edge [bend left] node {\tt Free} (Up);
    \draw[->] (Allocated) edge [bend left] node {\tt Measure} (Down);
  \end{tikzpicture}
  \caption{The FSM for the resource management phase}
  \label{fig:link-state-machine}
\end{figure}


%%% Local Variables:
%%% mode: japanese-latex
%%% TeX-master: "../bthesis"
%%% End:
