\chapter{Introduction}
\label{introduction}

\section{Background}
\label{introduction:background} 

The recent development of quantum technologies such as quantum computing, quantum networking and quantum sensing are expected to provide new capabilities. 
For example, quantum processors can theoretically simulate quantum systems whose size are intractable even for their classical equivalence.
The key to realize these new applications is quantum effect, such as superposition and entanglement, both of which cannot be observed in the classical world.

However, there are two major problems for transmitting quantum data to a distance location, which is required in certain situations such as distributed quantum computing.
One is "non-cloning theorem", which is the fact that quantum state cannot be copied. Unlike classical network, it is almost impossible to neither amplify a quantum state or send it forward because the quantum state will be heavily corrupted by the high probability of loss and high error rate.
The other problem is that it is so difficult to establish a bell pair between nodes separated by a long distance, again due to a photon will be spoiled by the physical noise and photon loss.

These two problems can be solved by using particular type of nodes called quantum repeaters. Quantum repeaters perform entanglement swapping and purification, each of which extends two neighboring bell pairs to a single longer bell pair, and improves the fidelity of the bell pair, respectively.
These operations end up with generating an end-to-end bell pair that can be used by quantum teleportation, which is the protocol to send an arbitrary quantum state to a distant location. 

Entanglement swapping and purification involve requires frequent message exchange with neighboring nodes in order to coordinate actions, such as entanglement swapping and purification, with neighboring nodes and those communication slow down the generation of an end-to-end bell pair.
However, a communication protocol called RuleSet-based communication protocol solves this problem by distribute an object called RuleSet, which a sequence of operations execute to each node. This feature reduces the amount of unnecessary communication and improves the scalability of the entire network.

\section{Research Contribution}
\label{introduction:research-contribution}

Multiple connections should be established simultaneously in order to enhance the overall performance and robustness of the entire network and the same thing can be applied to quantum network. 
However, the previous work only proposes the method to allocate required physical bell pairs and establish a single end-to-end bell pair, in other word, an single connection by consuming those physical resources.
This thesis proposes a protocol to realize three important tasks, which are the negotiation about what set of connections are going to be established, the one about when to switch from those in the previous round, and coordinated resource management between two nodes connected by each link.
It also discusses the updated procedure of establishing a new connection and tearing down one of the existing connections while several connections are being established by applying the proposed protocol.
The proposed approach in this thesis is validated by the simulation of RuleSet-based quantum networks under several circumstances.

\section{Thesis Structure}
\label{introduction:thesis-structure} 
% 本論文における以降の構成は次の通りである.

% ~\ref{background}章では,背景を述べる.
% ~\ref{issue}章では,本研究における問題の定義と,解決するための要件の整理を行う.
% ~\ref{proposed}章では,本研究の提案手法を述べる.
% ~\ref{implementation}章では,~\ref{proposed}章で述べたシステムの実装について述べる.
% ~\ref{evaluation}章では,\ref{issue}章で求められた課題に対しての評価を行い,考察する.
% ~\ref{conclusion}章では,本研究のまとめと今後の課題についてまとめる.


%%% Local Variables:
%%% mode: japanese-latex
%%% TeX-master: "../thesis"
%%% End:
