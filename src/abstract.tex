Abstract of master's Thesis - Academic Year 2023
\begin{center}
\begin{large}
\begin{tabular}{|p{0.97\linewidth}|}
    \hline
      \etitle \\
    \hline
\end{tabular}
\end{large}
\end{center}

~ \\
 Quantum networking is the new paradigm of networking that allows to send a quantum state and achieve various new applications.  RuleSet-based communication protocol is known to be one of the practical communication protocols to establish a scalable quantum network.
 Ideally, the network should be able to handle the multiple connections and the subsequence management of the link-level Bell pairs in order to improve the overall performance and the aggregate use of the available resource on the network, even when the number of the active connections changes. 
 However, the current state of the protocol only allows the establishment of a single connection and lacks the ability to terminate it and release the allocated resources on the link level.
 This thesis proposes the link management protocol for quantum network that allows the allocation and release of Bell pairs on the link level, which is the key to change of the number of active connections.
 It discusses the relationship between the connection management and the resource management in the link level. The behavior of the proposed protocol is validated by a set of numerical simulations on an existing quantum networking simulator.
~ \\
Keywords : \\
\underline{1. Quantum Networking},
\underline{2. RuleSet-Based Communication Protocol},
\underline{3. Networking Protocol},
\begin{flushright}
\edept \\
\eauthor
\end{flushright}
