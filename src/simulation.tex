\chapter{Simulation}
\label{simulation}

\section{QuISP (Quantum Internet Simulation Package)}

\subsection{Overview}

QuISP \cite{satoh2022quisp} is a quantum network simulator which aims to simulate the behavior of a large-scale quantum network. It is built on top of OMNeT++ \cite{10.5555/1416222.1416290}, which is an event-driven network simulator.
The reason why QuISP is built on top of OMNet++ is that OMNet++ allows users to define their own networking layers.
QuISP can simulate various types of errors, not only Pauli X error, Pauli Y error and Pauli Z error, but also relaxation error and excitation error.
Physical noise on an actual quantum system with $n$ qubits are usually simulated in the form of a density matrix, which would includes $2^n \times 2^n$ elementss and soon becomes intractable as $n$ becomes larger.
QuISP realizes the scalable simulation of quantum network by simulating the physical error using an error probability vector, which would take the following form.

\begin{equation}
  \overrightarrow{\pi}(t) = (\pi_I, \pi_X, \pi_Y, \pi_Z, \pi_R, \pi_E, \pi_L)
\end{equation}
It contains $m+1$ elements (m is the number of simulated error types)
The time evolution of error probability vector is provided by a transition error matrix $Q$.
\begin{equation}
  \overrightarrow{\pi}(t) = \overrightarrow{\pi}(t-1)Q 
\end{equation}

The error probability vector above is the one for a single qubit, so the one for $N$ qubit system contains $N(m+1)$ elements.

\subsection{Hardware Components}

Communication between two quantum nodes is achieved by transmission of photons via an optical fiber, and the fiber is mocked by an object called quantum link.
QuISP supports three main link architecture. 

\subsubsection{Memory-Memory}
The first one is Memory-Memory that two nodes are directly connected via a quantum link and the Bell State Analyzer is equipped in the receiver node.

\subsubsection{Memory-Interface-Memory}
The second one is Memory-Interface-Memory. Both end nodes of a quantum link emits photons to Bell State Analyzer located in the middle. After they become entangled, all the measurements results and required operations are sent back to both nodes.

\subsubsection{Memory-Source-Memory}
The last one Memory-Source-Memory. All the entanglement pairs are both generated and sent from the source of entangled photonic pair states in the middle.

\subsection{Software Components}

\subsubsection{Connection Manager}

Connection establishment is done when connection manager at the Initiator nodes sends ConnectionSetupRequest to the Responder node and intermediate nodes sends additional information such as those about QNIC.
After that, the connection manager at the responder node sends ConnectionSetupResponse to each node along the path of the connection.

\subsubsection{Hardware Monitor}
Hardware monitor is the module that collects the information of a quantum link such as fidelity and generation rate and pass those information to the routing daemon and the connection manager.

\subsubsection{Bell Pair Store}
Bell pair store is the module that stores the entanglement pairs generated from a support node such as a Bell State Analyzer.

\subsubsection{Rule Engine}
Rule engine is the component that is in charge for executing the given RuleSets and monitor the conditions of physical qubits.

\subsubsection{Real-Time Controller}
Real-time controller is the component that is responsible for the initialization of physical components and coordination of the timing for photon emission.

\subsubsection{Routing Daemon}
Routing daemon is the component that generates and exchanges the routing table for quantum network interface card, or QNIC in short.

\section{Implementation}

\subsection{RuleEngine}
\subsection{Link Allocation Policy Negotiation}
\subsubsection{Link Allocation Timing Negotiation}
\subsubsection{Resource Allocation}


%%% Local Variables:
%%% mode: japanese-latex
%%% TeX-master: "../bthesis"
%%% End:
