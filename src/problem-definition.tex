\chapter{Problem Definition}
\label{problem-definition}

\section{Problem Definition}

In order to maximize the overall performance and the aggregative use of resources in the entire network, it is desirable for several connections to be established in the real-time fashion.
However, there are two major obstacles to overcome in the case of a quantum network. 

One is the absence of a link management protocol for quantum network. There is previous work \cite{aparicio2011multiplexing} that proposes and compares the performance of various multiplexing strategies, but it does not mention any concrete methods to establish multiple connections and allocate the available physical links to each of these connections.

The other one is the lack of interaction between connection management and the subsequent resource management. The current RuleSet-based communication protocol \cite{matsuo2019quantum} only proposes the scheme to establish a single connection and it does not explain the method to tear it down and free the allocated physical links after the end of RuleSet execution. 

This thesis tackles the first problem by proposing the link management protocol the involves the negotiation about the set of connections to establish and the one about when to start the establishment. It also discuss the messages and their properties that are required to run this protocol.

Additionally, this thesis explains how the link management scheme is going to be triggered when a new connection is established and the old one is torn down. This explanation includes the methods to implement in the relevant software components when RuleSet-based quantum network is simulated or deployed in the real world.

\section{Assumptions}

This protocol is proposed based on the following assumptions.

\begin{itemize}
  \item The behavior of the entire network is determined by the collective decision made by each node.
  \item Each node allows the increase or decrease of the number of ongoing connections and does not refuse the setup of the new connection.
  \item The Bell pairs that are allocated but not used will be reallocated to one of the new connections.
  \item The link management protocol is triggered by the notification of either the setup of a new connection, or the teardown of an existing connection.
  \item The same amount of Bell pairs will be allocated to each RuleSet.
\end{itemize}

\section{Requirements}

This protocol has several requirements as follows.

\subsection{Functional requirements}

\subsubsection{Common}

\begin{itemize}
  \item Two neighboring nodes MUST agree on the next set of RuleSets (which is called the link allocation policy) and its order.
\end{itemize}

\subsubsection{Connection Setup}

\begin{itemize}
  \item Two neighboring nodes MUST also agree with when the new policy will be applied.
  \item If these negotiations are successful, available Bell pairs on the link level MUST be allocated to one of the RuleSets in the next link allocation policy.
\end{itemize}

\subsubsection{Connection Teardown}
\begin{itemize}
\item The execution of the old RuleSet MUST be terminated.
\item Link Bell pairs that are allocated but not consumed MUST be released from the RuleSet after its connection is terminated.
\end{itemize}


%%% Local Variables:
%%% mode: japanese-latex
%%% TeX-master: "./thesis"
%%% End:
