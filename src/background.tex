\chapter{Background}
\label{background}

\section{Quantum State}
\subsection{Quantum Bit}
A classical bit has two different states, which are 0 and 1.   Instead, those of a quantum bit (or \textbf{qubit} in short) are $|0\rangle$ and $|1\rangle$, each of which can be described as a vector. For example  
 $$|0\rangle = \Big[
\begin{array}{c}
1 \\
0 \\
\end{array}
\Big]
$$

$$|1\rangle = \Big[
\begin{array}{c}
0 \\
1 \\
\end{array}
\Big]$$

The state of a single 	qubit $|\psi\rangle$ can be described as follows.
$$ |\psi\rangle = \alpha |0\rangle + \beta |1\rangle \,(\alpha, \beta \in \mathbb{C}, |\alpha|^2+|\beta|^2=1)$$.
 After the operation called measurement, the quantum state would be collapsed into either 0 or 1.  The measurement probability of 0 is $|\alpha|^2$ and that of 1 is $|\beta|^2$. In other words, a single qubit can take both states probabilistically at the same time.  For instance, a qubit can be 
 
 \begin{equation}
	|\psi\rangle = \frac{1}{\sqrt{2}}|0\rangle + \frac{1}{\sqrt{2}}|1\rangle \tag{1}
\end{equation}

 which can be 50\% 0 and 50\% 1.

 \subsection{Bloch sphere}
 Because $|\alpha|^2 + |\beta|^2 = 1$, the notation of a single qubit state can be represented like this.

\begin{equation}
|\psi\rangle = e^{i\gamma} (\cos{\frac{\theta}{2}} + e^{i\phi} \sin{\frac{\theta}{2}}) (\gamma, \phi, \theta \in \mathbb{R})
\end{equation}.

Because $e^{i\gamma}$ is just a global state, it can be ignored and the same state can be rewritten like this.

\begin{equation}
 |\psi\rangle =  \cos{\frac{\theta}{2}} + e^{i\phi} \sin{\frac{\theta}{2}} (\phi, \theta \in \mathbb{C})
\end{equation}

Because the equation above has two parameters,  any pure single qubit state can be considered as a point on the surface and its geometric representation is called \textbf{Bloch sphere}.

\begin{figure}[ht]
  \centering
  \tikz{
    \tikzstyle{st}=[lightgray, fill, fill opacity=0.2];
    \coordinate(o)at(0,0); 
    \draw(o)circle(2cm); 
    \draw[fill](o)circle(1.5pt);%origin
    \draw[st](o)--(56.7:0.4)arc(56.7:90.:0.4)--cycle;%theta angle
    \draw(0.18,0.6)node{$\theta$};
    \draw[st](o)--(-135.7:0.4)arc(-135.7:-33.2:0.4)--cycle;%varphi angle
    \draw(0.14,-0.58)node{$\varphi$};
    \draw[->](o)--(-0.81,-0.79) node[above left]{\ $x$};%x
    \draw[->](o)--(2,0)node[right]{$y$};%y
    \draw[->](o)--(0,2)node[below right]{$z$}node[above]{\ $\ket{0}$};%z |0>
    \draw[rotate around={0.:(0.,0.)},dashed](0,0)ellipse(2cm and 0.9cm);%ellipse
    \draw[thick,->](o)--(0.70,1.07)node[above]{\ $\ket{\psi}$};%state vector
    \draw[densely dotted,->](o)--(0,-2)node[below]{\ $\ket{1}$};%-z |1>
    \draw[dotted](o)--(0.7,-0.46)--(0.7,1);%triangle
  }
    
\newpage
\caption{Bloch Sphere}
\end{figure}

\subsection{Multi-Qubit State}
  The quantum state for multi-qubits is a \textbf{tensor product} of a state vector of each qubit.  The general notation of two qubit state is
  
\begin{flalign}
    |\psi\rangle & = (\alpha |0\rangle + \beta |1\rangle) \otimes  (\gamma |0\rangle + \delta |1\rangle) \\
    & = \alpha \gamma |00\rangle + \alpha \delta |01\rangle + \beta \gamma |10\rangle + \beta \delta |11\rangle \\ 
   & (\alpha, \beta, \gamma, \delta \in \mathbb{C}, |\alpha|^2+|\beta|^2+|\gamma|^2+|\delta|^2=1)
 \end{flalign}.
  
  For example, the state $|00\rangle$ is equal to 
  
\begin{equation}
  \Big[
\begin{array}{c}
1 \\
0 \\
\end{array}
\Big]
\otimes
 \Big[
\begin{array}{c}
1 \\
0 \\
\end{array}
\Big]
= \Big[
\begin{array}{c}
1 \\
0 \\
0 \\
0 \\
\end{array}
\Big]
\end{equation}.

 However, some quantum states such as
 
 \begin{equation}
 	|\psi\rangle = \frac{1}{\sqrt{2}}|00\rangle + \frac{1}{\sqrt{2}}|11\rangle
 \end{equation}
 
 cannot be decomposed into quantum state of each qubit.  These special quantum states are called \textbf{entangled} states.

\section{Quantum Operations}
\subsection{I gate}

I gate is equal to the 2x2 identity matrix, which is 

\begin{equation}
I = \begin{bmatrix}
1 & 0 \\
0 & 1 \\
\end{bmatrix}
\end{equation}.

For example,

\begin{equation}
 I|0\rangle = \begin{bmatrix}
1 & 0 \\
0 & 1 \\
\end{bmatrix} 
\left[
\begin{array}{c}
1 \\
0 \\
\end{array}
\right]
= \left[
\begin{array}{c}
1 \\
0 \\
\end{array}
\right]
= |0\rangle
\end{equation}

\begin{equation}
I|1\rangle = \begin{bmatrix}
1 & 0 \\
0 & 1 \\
\end{bmatrix} 
\left[
\begin{array}{c}
0 \\
1  \\
\end{array}
\right]
= \left[
\begin{array}{c}
0 \\
1 \\
\end{array}
\right]
= |1\rangle
\end{equation}.

\subsection{X Gate}
\subsubsection{X gate}

X gate flips the logical value of a qubit.

\begin{equation}
X = \begin{bmatrix}
0 & 1 \\
1 & 0 \\
\end{bmatrix}
\end{equation}.

For example,
\begin{equation}
X|0\rangle = \begin{bmatrix}
0 & 1 \\
1 & 0 \\
\end{bmatrix} 
\left[
\begin{array}{c}
1 \\
0 \\
\end{array}
\right]
= \left[
\begin{array}{c}
0 \\
1 \\
\end{array}
\right]
= |1\rangle
\end{equation}

\begin{equation}
 X|1\rangle = \begin{bmatrix}
0 & 1 \\
1 & 0 \\
\end{bmatrix} 
\left[
\begin{array}{c}
0 \\
1  \\
\end{array}
\right]
= \left[
\begin{array}{c}
1 \\
0 \\
\end{array}
\right]
= |0\rangle
\end{equation}.

\subsection{Y gate}

Y gate flips the logical value of a qubit and add an imaginary number.

\begin{equation}
 Y = \begin{bmatrix}
0 & -i \\
i & 0 \\
\end{bmatrix}
\end{equation}.

For example,
\begin{equation}
Y|0\rangle = \begin{bmatrix}
0 & -i \\
i & 0 \\
\end{bmatrix} 
\left[
\begin{array}{c}
1 \\
0 \\
\end{array}
\right]
= \left[
\begin{array}{c}
0 \\
i \\
\end{array}
\right]
= i|1\rangle
\end{equation}

\begin{equation}
Y|1\rangle = \begin{bmatrix}
0 & -i \\
i & 0 \\
\end{bmatrix} 
\left[
\begin{array}{c}
0 \\
1  \\
\end{array}
\right]
= \left[
\begin{array}{c}
-i \\
0 \\
\end{array}
\right]
= -i|0\rangle
\end{equation}.

\subsection{Z Gate}
Z gate flips the phase of $ |1\rangle$

\begin{equation}
 Z = \begin{bmatrix}
1 & 0 \\
0 & -1 \\
\end{bmatrix}
\end{equation}.

For example,
\begin{equation}
 Z|0\rangle = \begin{bmatrix}
1 & 0 \\
0 & -1 \\
\end{bmatrix} 
\left[
\begin{array}{c}
1 \\
0 \\
\end{array}
\right]
= \left[
\begin{array}{c}
1 \\
0 \\
\end{array}
\right]
= |0\rangle
\end{equation}

\begin{equation}
Z|1\rangle = \begin{bmatrix}
1 & 0 \\
0 & -1 \\
\end{bmatrix} 
\left[
\begin{array}{c}
0 \\
1  \\
\end{array}
\right]
= \left[
\begin{array}{c}
0 \\
-1 \\
\end{array}
\right]
= -|1\rangle
\end{equation}.

\subsection{H Gate}
H gate creates superposition.
\begin{equation}
 H = \frac{1}{\sqrt{2}}\begin{bmatrix}
1 & 1\\
1 & -1 \\
\end{bmatrix}
\end{equation}.

For example,
\begin{equation}
H|0\rangle = \frac{1}{\sqrt{2}}\begin{bmatrix}
1 & 1\\
1 & -1 \\
\end{bmatrix}\left[
\begin{array}{c}
1 \\
0 \\
\end{array}
\right]
= \frac{1}{\sqrt{2}} \left[
\begin{array}{c}
1 \\
1 \\
\end{array}
\right]
= \frac{1}{\sqrt{2}} (|0\rangle + |1\rangle)
\end{equation}

\begin{equation}
H|1\rangle = \begin{bmatrix}
1 & 1\\
1 & -1 \\
\end{bmatrix} 
\left[
\begin{array}{c}
0 \\
1  \\
\end{array}
\right]
= \frac{1}{\sqrt{2}} \left[
\begin{array}{c}
1 \\
-1 \\
\end{array}
\right]
=\frac{1}{\sqrt{2}} (|0\rangle - |1\rangle)
\end{equation}.

\subsection{CNOT Gate}
A CNOT gate involves two qubits, one is called \textbf{controlled qubit} and the other is called \textbf{target qubit}.  If the controlled qubit is 1, the bit value of the target qubit is flipped.

\begin{equation}
CNOT = \begin{bmatrix}
1 & 0 & 0 & 0 \\
0 & 1 & 0 & 0 \\
0 & 0 & 0 & 1 \\
0 & 0 & 1 & 0 \\
\end{bmatrix}
\end{equation}.

For example,
\begin{equation}
CNOT_{0,1}|10\rangle = 
\begin{bmatrix}
1 & 0 & 0 & 0 \\
0 & 1 & 0 & 0 \\
0 & 0 & 0 & 1 \\
0 & 0 & 1 & 0 \\
\end{bmatrix}
 \left[
\begin{array}{c}
0 \\
0 \\
1 \\
0 \\
\end{array}
\right]
=  \left[
\begin{array}{c}
0 \\
0 \\
0 \\
1 \\
\end{array}
\right] 
= |11\rangle 
\end{equation}

\begin{equation}
CNOT_{0,1}|11\rangle = 
\begin{bmatrix}
1 & 0 & 0 & 0 \\
0 & 1 & 0 & 0 \\
0 & 0 & 0 & 1 \\
0 & 0 & 1 & 0 \\
\end{bmatrix}
 \left[
\begin{array}{c}
0 \\
0 \\
0 \\
1 \\
\end{array}
\right]
=  \left[
\begin{array}{c}
0 \\
0 \\
1 \\
0 \\
\end{array}
\right] 
= |10\rangle 
\end{equation}.

\subsection{Measurement}
Quantum measurement can be described by using a group of measurement operators $\{M_m\}$
($m$ is the measurement result that is expected to get).
 If the quantum state before measurement is $|\psi\rangle$, the measurement probability of value $m$ is 
 $$p(m) = \langle \psi|M^{\dagger}_m M_m|\psi\rangle$$

 The quantum state after the measurement is 
 $$\frac{M_m|\psi\rangle}{\sqrt{\langle \psi|M^{\dagger}_m M_m|\psi\rangle}}$$

The measurement operators satisfy the completeness equation
$$\sum_{m} M^{\dagger}_m M_m = I$$

Also, the sum of the measurement probability of each possible measurement outcome is equal to one.
$$\sum_{m} p(m) = \langle \psi|\sum_{m} M^{\dagger}_m M_m|\psi\rangle = 1$$

\section{Quantum Circuit}
Here is the example of a quantum circuit.

\begin{figure}[ht]
  \begin{center}
    \begin{tikzpicture}
    \begin{yquant}
      qubit {$\ket{\reg_{\idx}}$} a[3];
      h a[0];
      h a[1];
      h a[2];
      cnot a[0] | a[1];
      x a[2];	
      y a[0];	 
      cnot a[1] | a[2]; 
      measure a[0-2];					 
     \end{yquant}
  \end{tikzpicture}
\caption{A example of quantum circuit}
\end{center}
\end{figure}

Each horizontal line represents each qubit and the square boxes that contain alphabets mean single quantum gates.  The sign which involves a vertical line means a CNOT gate, and the box on the most right side indicates measurement. 

\section{Quantum Entanglement}

Quantum entanglement is a special type of quantum state that cannot be described in the form of tensor product of the state of each particle.

\subsection{Bell Pair}
The entangled states between two qubits are called bell pairs, and each of four states has a special notation.

\begin{equation}
  |\Phi^+\rangle = \frac{|00\rangle + |11\rangle}{\sqrt{2}}
  \end{equation}
  
  \begin{equation}
 |\Phi^-\rangle = \frac{|00\rangle - |11\rangle}{\sqrt{2}}
 \end{equation}
 
 \begin{equation}
 |\Psi^+\rangle = \frac{|01\rangle + |10\rangle}{\sqrt{2}}
 \end{equation}
 
 \begin{equation}
  |\Psi^-\rangle = \frac{|01\rangle - |10\rangle}{\sqrt{2}}
  \end{equation}.

\subsection{Multipartite Entanglement}
There are cases that more than two qubits are entangled and that state is called Greenberger–Horne–Zeilinger state or GHZ state.

Here is the braket notation of the GHZ state that involves three qubits.
\begin{equation}
  |GHZ\rangle = \frac{|000\rangle + |111\rangle}{\sqrt{2}}
\end{equation}.

In the general case, the braket notation of the GHZ state of N qubits is the following.
\begin{equation}
  |GHZ\rangle = \frac{|0\rangle^{\otimes N} + |1\rangle^{\otimes N}}{\sqrt{2}}
\end{equation}.

\subsection{Bell State Measurement}
Bell state measurement is a special type of quantum measurement that determines which bell pair the given two qubit entangled state is.

\begin{figure}[ht]
  \begin{center}
    \begin{tikzpicture}
    \begin{yquant}
      qubit {$\ket{\reg_{\idx}}$} a[2];
      cnot a[0] | a[1];
      h a[0];	 
      measure a[0-1];					 
     \end{yquant}
  \end{tikzpicture}
\caption{Quantum circuit for bell state measurement}
\end{center}
\end{figure}

\begin{tabular}{|l|r|} \hline
  Measurement results & Bell state \\ \hline \cline{1-2}
  00 &  $|\Phi^+\rangle$ \\ \cline{1-2}
  01 & $|\Phi^-\rangle$ \\  \cline{1-2}
  10 &  $|\Psi^+\rangle$ \\ \cline{1-2}
  11 & $|\Psi^-\rangle$ \\  \hline  \cline{1-2}
\end{tabular}

\subsection{Quantum Teleportation}

Unlike classical communication, quantum states cannot be just copied and transmit to other nodes due to the no-cloning theorem, which forbids duplication of any quantum state.  However, a method called quantum teleportation \cite{teleportation} was proposed, which overcomes the restriction and allows sender to transmit single qubit state to a distant location. 
 		
This method requires both the single qubit state and a new Bell pair, and also the sender have to prepare two qubits and the receiver have to prepare one qubit.  After applying a CNOT gate and an H gate in the figure above, the sender have to measure both qubits and send those measurement results over the classical network.  After the receiver get those measurement results and apply some quantum gates if the measurement results of corresponding qubits on the sender's side are 1, in order to correct on the quantum state on the receiver's side.

\begin{figure}[ht]
  	\begin{center}
  		\begin{tikzpicture}
			\begin{yquant}
	 			qubit {$\ket{\reg_{\idx}}$} a[2];
	 			qubit {$\ket{\reg_{\idx}}$} b[1];
	 			h a[1];
        cnot a[1] | b[0];
	 			cnot a[0] | a[1];
        h a[0];	 
        measure a[0-1];	
        z b[0] | a[0];	
        x b[0] | a[1];				 
 			\end{yquant}
		\end{tikzpicture}
	\caption{Quantum circuit for quantum teleportation}
	\end{center}
\end{figure}

\subsection{Entanglement Swapping}
\begin{figure}[ht]
  \begin{center}
    \begin{tikzpicture}
    \begin{yquant}
       qubit {$\ket{\reg_{\idx}}$} a[1];
       qubit {$\ket{\reg_{\idx}}$} b[2];
      qubit {$\ket{\reg_{\idx}}$} c[1];
       h a[0];
      h c[0];
       cnot a[0] | b[0];
       cnot c[0] | b[1];
      cnot b[0] | b[1];
      h b[0];	 
      measure b[0-1];	
      z a[0] | b[0];	
      x c[0] | b[1];		 
     \end{yquant}
  \end{tikzpicture}
\caption{Quantum circuit for entanglement swapping}
\end{center}
\end{figure}

\subsection{Entanglement Purification}

\section{Quantum Networking}
\subsection{Quantum Node}
\subsection{Quantum Repeater}
\subsection{Quantum Link}
\subsection{Major Applications of Quantum Networking}

